\chapter{\Edgeworth{} Formative Interview Protocol}
\label{app:edgeworth-formative-protocol}

\section{Introduction}

\begin{itemize}
    \item Today I'd like to first learn a bit about how you use and/or author diagrams for your work. 
    \item Is it okay for me to record this session? The recording will only be shared within the team for research purposes.
\end{itemize}

\section{Needs and Requirements}

\begin{itemize}
    \item \textbf{What do you teach?} (subject, students, institution)
    \item \textbf{Do you use diagrams in your instructional content?} (slides, handouts, problem sets, exams)
    \item \textbf{What visual problems do you want to make?}
    \begin{itemize}
        \item Can you show me some examples of visual problems you have created?
        \item Do you make families of related problems? How many instances in each family? Why?
        \item What are the diagram types?
        \begin{itemize}
            \item Do you have different problem types within these types of diagrams?
        \end{itemize}
        \item What learning goal is served by the problem as a whole?
        \begin{itemize}
            \item What role does the diagram play?
            \item Do students manipulate or create diagrams as part of solving problems? If so, can you give an example? If not, would you like to develop problems where they could manipulate or create a diagram?
            \item What is the relationship between instructional material diagrams and problem diagrams?
        \end{itemize}
    \end{itemize}
    \item \textbf{Do students learn about graphical representations, like charts and graphs of functions in your classes?}
    \begin{itemize}
        \item If so, are they tested on them? Do they do exercises that contain diagrammatic contents (problems with pictures as a part of the prompt, or a part of the answer)?
        \item Are students expected to participate in drawing in class, either physically or digitally?
    \end{itemize}
\end{itemize}

\section{Tooling}

\begin{itemize}
    \item How do you create and maintain your diagrams?
    \begin{itemize}
        \item What are the tools involved?
        \item What are the barriers you encountered when creating or maintaining these diagrams?
        \item How did you solve some, if any, of your problems caused by these barriers?
    \end{itemize}
    \item Does your tool help maintain the relationship between prose/symbolic prompt and diagram?
\end{itemize}

\section{Authoring Process}

\begin{itemize}
    \item What is the process for creating new problems?
    \item How are the variations designed?
    \item Do students generate diagrams?
\end{itemize}

\section{Meta}

\begin{itemize}
    \item What are your ideal problem types?
    \begin{itemize}
        \item What's missing in E-learning platforms?
        \item What's missing from the theory of learning?
    \end{itemize}
    \item What (pedagogically useful) problem types are not created because of tool limitations?
    \item What diagram/problem types appear in other physical mediums (e.g., textbooks) but not in e-learning platforms?
\end{itemize}