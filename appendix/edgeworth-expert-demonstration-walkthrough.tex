\chapter{\Edgeworth{} Expert Walkthrough Demonstration Protocol}
\label{app:expert-walkthrough}

\section*{Demographics}
\begin{itemize}
    \item Is it okay for me to record this session? The recording will only be shared within the team for research purposes.
    \item What do you do professionally?
    \item How many years have you been doing this?
\end{itemize}

\section*{Demo (2 mins)}
\subsection*{Setup}
You are creating sets of translation problems, i.e. problems asking students to determine diagrammatic instances and noninstances of a textual description, using Edgeworth.

\textbf{Share screen and perform this demonstration:}
\begin{itemize}
    \item Select a problem (from another domain) from the dropdown.
    \item Click ``Generate Variations''.
    \item Click checkboxes to collect diagrams.
    \item Click ``Show Problem'' to see the problem.
    \item Click ``Hide Problem'' to go back.
    \item Drag ``Number of Variations'' to increase the number of diagrams on the grid.
\end{itemize}

\section*{Definitions}
\begin{itemize}
    \item \textbf{Identical to original}: the diagram is identical (modulo layout differences) to the original.
    \item \textbf{Correct}: the diagram is a correct answer to the prompt.
    \item \textbf{Incorrect}: the diagram is not a correct answer to the prompt.
    \item \textbf{Blatantly incorrect}: the diagram shows a blatantly incorrect answer to the prompt. Students who have a cursory understanding of the relevant geometry concept can identify the diagram as an incorrect one.
\end{itemize}

\section*{Individual diagram ratings / problem generation}
For a sample of 5-10 prompts, rate each generated diagram as one of: identical to original, correct, incorrect, and blatantly incorrect. You'll eventually use these diagrams to create a problem.

\textbf{After rating the diagrams, optionally answer the following questions on some of the ratings:}
\begin{itemize}
    \item Why did you rate the diagram as ``<option>''?
    \item What needs to be changed to turn it into ``<another-option>''?
    \item How does the rating relate to your instructional goals?
\end{itemize}

\section*{Problem creation}
Now, create the best problem you can make from the pool of diagrams:
\begin{itemize}
    \item Choose 4 answers; no limitations on the number of correct or incorrect answers.
    \item If needed, you can generate more than 10 diagrams using the slider on the left.
\end{itemize}

\section*{Problem quality}
For each of the problems authored, answer the following questions:
\begin{itemize}
    \item \textbf{Ecological validity}: Would you use this problem in your course?
    \begin{itemize}
        \item In what context would you include it (e.g., in class, quizzes, problem sets, exams)?
        \item What did you like about the problem?
        \item How can it be improved? 
        \begin{itemize}
            \item NOTE: try to uncover if the issue is with the diagram appearance versus content.
        \end{itemize}
        \item How similar is this problem to problems you've seen in the past?
    \end{itemize}
    \item \textbf{Authoring effort}: If you were going to create this problem, how would you do it?
    \begin{itemize}
        \item How much work do you expect the authoring to take?
        \begin{itemize}
            \item In terms of time?
            \item In terms of effort?
        \end{itemize}
        \item How often do you author problems like this?
    \end{itemize}
\end{itemize}