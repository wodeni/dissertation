\chapter{Diagrammer Interview Codebook}
\label{app:interview-codebook}

This codebook categorizes various aspects related to the process of creating diagrams, the reuse of diagram components, adoption of tools, and individual abilities. Each section breaks down these aspects into themes and subcategories based on observed behaviors and statements during interviews with domain experts, described in \cref{chp:interviews}.

\begin{itemize}
    \item \textbf{Personal Background}
    \begin{itemize}
        \item Gender: Gender of the interviewee.
        \item Occupation: Professional role of the interviewee.
        \item CMU / Non-CMU: Whether the interviewee is affiliated with Carnegie Mellon University.
        \item American / Non-American: Nationality or cultural background of the interviewee.
    \end{itemize}

    \item \textbf{Reusability}
    \begin{itemize}
        \item Starting from an Existing Diagram: Modifying an existing file or project, created either by oneself or another, to develop a new diagram.
        \item Reuse Lower-Level Components: Reuse basic components such as shapes, colors, and properties of the diagram.
        \begin{itemize}
            \item Representation Reuse: Coded as reusing shapes and properties in a new context.
        \end{itemize}
        \item Reuse Representation Concepts: Reusing established mappings, such as ``vector as arrow,'' to maintain visual consistency across diagrams.
        \item Version Control in Diagramming (DM) Tools: ability to manage versions and return to previous versions within a diagramming tool.
        \item Reuse Across Projects: Using the same diagram in multiple contexts, such as different research papers.
        \item Version Control in Programming Language (PL) Tools: as with diagramming tools, enabling prior-version management (coded together with DM version control in this system).
    \end{itemize}

    \item \textbf{Using Multiple Tools for Diagram Creation}
    \begin{itemize}
        \item For Vector Graphics: Converting output from one tool to a vector format (e.g., SVG) for scalability.
        \item For Annotation: Importing a partially complete diagram into another tool for adding annotations (e.g., arrows, text, LaTeX labels) without modifying the original content.
        \item For Accurate Rendering: Generating base data or components in one tool and importing them into another for final diagram completion, especially in cases involving data or physical phenomena.
    \end{itemize}

    \item \textbf{Diagramming Process}
    \begin{itemize}
        \item Choosing an Appropriate Representation: Surveying existing representations or designing new mappings from abstract objects to visual elements (e.g., vector to arrow).
        \item Iterative Process: Returning to prior versions or generating different versions of a representation by tweaking configurations or style (e.g., layout, color).
        \begin{itemize}
            \item Instance: A unique version of a representation (e.g., a red arrow vs. a black arrow).
        \end{itemize}
        \item Sketching: Initial, low-fidelity diagram creation as a preliminary step.
        \item Other: Additional notes on the process, if any.
    \end{itemize}

    \item \textbf{Adoption of New Tools}
    \begin{itemize}
        \item Interest in New Tools: Exploring new tools, even without full commitment to usage.
        \begin{itemize}
            \item ``I Should Try This Tool'': Intent to try but haven’t yet.
            \item ``I Tried and Didn’t Like It'': Trialed but not adopted.
        \end{itemize}
        \item Adoption Reasons: Objective or situational motivations for trying a tool, such as advisor recommendation.
        \item Tool Rationale: Specific advantages for adopting a tool based on task needs or types.
    \end{itemize}

    \item \textbf{Precision in Diagram Creation}
    \begin{itemize}
        \item Feedback Cycle: Processes or tools that facilitate feedback during diagramming.
        \item Selection: precision of shape/object selection in the user interface.
    \end{itemize}

    \item \textbf{Learning Methods for Visualization Tools}
    \begin{itemize}
        \item Reading the Manual: Formal study of tool documentation.
        \item Trial-and-Error: Learning through experimentation.
        \item Searching Online: Seeking guidance from diverse online sources.
    \end{itemize}

    \item \textbf{Specific Tools}
    \begin{itemize}
        \item Pros and Cons: Evaluations of tools, often based on interviewee feedback.
        \item Criteria for ``Good'' Diagrams: Characteristics defining quality or effectiveness in diagrams.
    \end{itemize}

    \item \textbf{Self-Reported Abilities}
    \begin{itemize}
        \item Hand-Drawing Ability: Interviewees’ perception of their drawing skills.
        \begin{itemize}
            \item Good at Drawing: High confidence in drawing skills.
            \item Not Good at Drawing: Low confidence, with possible understatements due to humility.
        \end{itemize}
        \item Programming Ability: Skill level in programming as it pertains to diagramming.
        \item Diagram Types: Preference for data-intensive (empirical) vs. conceptual diagrams.
    \end{itemize}

    \item \textbf{Tool Preferences}
    \begin{itemize}
        \item Direct Manipulation vs. Programmatic: Preference for hands-on tool usage versus code-driven diagramming.
    \end{itemize}

    \item \textbf{Diagram Frequency}
    \begin{itemize}
        \item Number of Diagrams Made: Volume of diagram creation, providing insight into experience level.
    \end{itemize}
\end{itemize}
