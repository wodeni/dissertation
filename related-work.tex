% Edgeworth 


\section{\Edgworth related work}
\label{sec:edgeworth-related}

\subsection{Using contrasting cases to improve representational fluency}

Representational fluency refers to the ability to quickly understand a visual representation and to use it to solve domain-specific tasks~\cite{multipleReps}. To become representationally fluent, an important first step is to identify meaningful aspects of a particular representation. \citet{perceptualLearning} show that mapping between symbolic and visual representations leads to intuitions about the way equivalent structures relate to each other. The learning that results from constructing connections between symbols and diagrams can be more flexible. Students are better at transferring their learning from the problems they have explicitly practiced to more open-ended problems and their conceptual understanding is better~\cite{25learning}. 

In addition to mapping between representations, \citet{samenessAndDifference} also showed that contrasting cases help students discern crucial parts of a particular representation. Early on, students benefit from discerning instances and noninstances that differ in only one dimension of variation. As students become more fluent, a \emph{fusion} of multiple varying dimensions in problems may be necessary~\cite{fusion}.

\subsection{Multiplicity of examples and problem generation}

In addition to training representational fluency, multiple examples and repeated, varied practice are well-documented strategies for broader learning goals in the learning science literature. Many studies have demonstrated substantial STEM learning benefits for multiple worked examples per topic \cite{PBB07}. Equally important is research indicating the importance of active learning \cite{CW14, DMM19} and repeated practice \cite{deliberatePractice, SSL98} that occurs within varied contexts \cite{PV94, RT07} and involves direct explanatory feedback \cite{perceptualLearning}.

As a result, a number of authoring tools exist for large-scale production of examples and practice problems. Intelligent Tutoring Systems (ITS) are automated curricula that often include worked examples and practice problems that are customized to individual students. The Cognitive Tutor Authoring Tools (CTAT) is an ITS authoring platform~\cite{CTAT}. CTAT has a ``Mass Production'' feature that lets the user create a problem template and insert problem-specific values via a spreadsheet~\cite{massProduction}. The ASSISTment builder allows authors to ``variablize” numerical values in problem templates for automatic generation~\cite{ASSISTment}. In the computer-aided education literature, a number of systems were also proposed for problem generation~\cite{compEduCACM, synthDeduction, synthGeometry} In both lines of work, most systems don’t tackle the problem of diagram generation—they mostly generate symbolic problems and examples. Notably, \citet{synthGeometry} generated ruler-and-compass geometry constructions automatically, which is a significantly narrower domain than what \Edgeworth targets. 