% Penrose

\subsection{Diagramming Systems}
\label{sec:RelatedSystems}

Here we consider how our system design relates to other systems that convert abstract mathematical ideas into visual diagrams.  Other classes of tools, such as general-purpose drawing tools (\eg{}, \emph{Adobe Illustrator}) can also be used to make diagrams, though one quickly runs into barriers, such as for large-scale diagram generation or evolving the style of a large collection of existing diagrams.  A broader discussion of related work can be found in a pilot study we did on how people use diagramming tools~\cite{Ni:2020:HDE}.

There are three main kinds of systems that convert an abstract form of input (\eg{}, an equation or code) into a visual representation.  Language-based systems, such as \textit{TikZ}~\cite{TikZ-Manual} (which builds on \TeX), are domain-agnostic and provide significant flexibility for the visual representation. Their use of ``math-like'' languages influenced the design of \Substance.  However, existing systems do not aim to separate mathematical content from visual representation. For instance, TikZ is domain- and representation-agnostic because it requires diagrams to be specified at a low level (\eg{}, individual coordinates and styles) making programs hard to modify or reuse.  Moreover, since there are only shallow mathematical semantics, it becomes hard to reason about programs at a domain level.

Plotting-based systems, like \emph{Mathematica} and \emph{GeoGebra}~\cite{Hohenwarter:2004:CDG} enable standard mathematical expressions to be used as input and automatically generate attractive diagrams.  Just as a graphing calculator is easy to pick up and use for most students of mathematics, these tools inspired us to provide a ``tiered'' approach to \Penrose{} that makes it accessible to users with less expertise in illustration (\figref{NoviceExpertUsers}). However, much like a graphing calculator, the visual representations in these systems are largely ``canned,'' and the set of easily accessible domains is largely fixed.  For instance, \emph{Mathematica} does not permit user-defined types, and to go beyond system-provided visualization tools, one must provide low-level directives (in the same spirit as tools like \textit{TikZ}).

Finally, systems like \emph{graphviz}~\cite{Graphviz}, and \emph{Geometry Constructions Language}~\cite{Janivcic:2006:GCLC} translate familiar domain-specific language into high-quality diagrams. Here again, the domains are fairly narrow and there is little to no opportunity to expand the language or define new visualizations.  Yet the convenience and power of such systems for their individual domains inspired us to build a system with greater extensibility. More broadly, while all these systems share some design goals with ours, a key distinction is that \Penrose\ is designed from the ground up as an extensible \emph{platform} for building diagramming tools, rather than a monolithic end-user tool.

% Edgeworth 
\section{\Edgworth related work}
\label{sec:edgeworth-related}

\subsection{Using contrasting cases to improve representational fluency}

Representational fluency refers to the ability to quickly understand a visual representation and to use it to solve domain-specific tasks~\cite{multipleReps}. To become representationally fluent, an important first step is to identify meaningful aspects of a particular representation. \citet{perceptualLearning} show that mapping between symbolic and visual representations leads to intuitions about the way equivalent structures relate to each other. The learning that results from constructing connections between symbols and diagrams can be more flexible. Students are better at transferring their learning from the problems they have explicitly practiced to more open-ended problems and their conceptual understanding is better~\cite{25learning}. 

In addition to mapping between representations, \citet{samenessAndDifference} also showed that contrasting cases help students discern crucial parts of a particular representation. Early on, students benefit from discerning instances and noninstances that differ in only one dimension of variation. As students become more fluent, a \emph{fusion} of multiple varying dimensions in problems may be necessary~\cite{fusion}.

\subsection{Multiplicity of examples and problem generation}

In addition to training representational fluency, multiple examples and repeated, varied practice are well-documented strategies for broader learning goals in the learning science literature. Many studies have demonstrated substantial STEM learning benefits for multiple worked examples per topic \cite{PBB07}. Equally important is research indicating the importance of active learning \cite{CW14, DMM19} and repeated practice \cite{deliberatePractice, SSL98} that occurs within varied contexts \cite{PV94, RT07} and involves direct explanatory feedback \cite{perceptualLearning}.

As a result, a number of authoring tools exist for large-scale production of examples and practice problems. Intelligent Tutoring Systems (ITS) are automated curricula that often include worked examples and practice problems that are customized to individual students. The Cognitive Tutor Authoring Tools (CTAT) is an ITS authoring platform~\cite{CTAT}. CTAT has a ``Mass Production'' feature that lets the user create a problem template and insert problem-specific values via a spreadsheet~\cite{massProduction}. The ASSISTment builder allows authors to ``variablize” numerical values in problem templates for automatic generation~\cite{ASSISTment}. In the computer-aided education literature, a number of systems were also proposed for problem generation~\cite{compEduCACM, synthDeduction, synthGeometry} In both lines of work, most systems don’t tackle the problem of diagram generation—they mostly generate symbolic problems and examples. Notably, \citet{synthGeometry} generated ruler-and-compass geometry constructions automatically, which is a significantly narrower domain than what \Edgeworth targets. 